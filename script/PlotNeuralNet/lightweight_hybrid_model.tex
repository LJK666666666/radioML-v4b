% Lightweight Hybrid Complex-ResNet Model Visualization
% This LaTeX file accurately represents the actual lightweight model architecture
% from build_lightweight_hybrid_model() function in hybrid_complex_resnet_model.py
% Key features: 1) Simple 3-residual-block architecture
%              2) ComplexResidualBlock(64) -> ComplexResidualBlock(128, strides=2) -> ComplexResidualBlockAdvanced(256, strides=2)
%              3) Pure complex processing throughout with final real conversion

\documentclass[border=15pt, multi, tikz]{standalone}
\usepackage{import}
\subimport{./layers/}{init}
\usetikzlibrary{positioning}

\begin{document}
\begin{tikzpicture}
\tikzstyle{connection}=[ultra thick,every node/.style={sloped,allow upside down},draw=\edgecolor,opacity=0.6]
\tikzstyle{copy_connection}=[ultra thick,every node/.style={sloped,allow upside down},draw={rgb:blue,4;red,1;green,1;black,3},opacity=0.6]

% 残差连接样式
\tikzstyle{residual_connection}=[ultra thick,every node/.style={sloped,allow upside down},draw=green!70!black,opacity=0.8,dashed]

% 1. 输入层
\pic[shift={(0,0,0)}] at (0,0,0) 
    {RightBandedBox={
        name=input,
        caption=输入,
        xlabel={{2,}},
        ylabel=128,
        fill=\ConvColor,
        height=35,
        width=2,
        depth=25
    }};

% 2. Permute层 (2,128) -> (128,2)
\pic[shift={(2,0,0)}] at (input-east) 
    {Box={
        name=permute,
        caption=Permute,
        xlabel={(128, 2)},
        fill=\ReshapeColor,
        height=25,
        width=2,
        depth=2
    }};

% 3. 初始复数卷积 + 池化
\pic[shift={(3,0,0)}] at (permute-east) 
    {RightBandedBox={
        name=complex_conv1,
        caption=ComplexConv1D,
        xlabel={{32,}},
        ylabel={{64,}},
        fill=\ConvColor,
        height=28,
        width=3,
        depth=15
    }};

% 批归一化和激活
\pic[shift={(1,0,0)}] at (complex_conv1-east) 
    {Box={
        name=complex_bn1,
        caption=BN+Act,
        fill=\ActivationColor,
        height=28,
        width=1,
        depth=15
    }};

% 复数池化
\pic[shift={(1,0,0)}] at (complex_bn1-east) 
    {Box={
        name=complex_pool1,
        caption=ComplexPool,
        xlabel={{32,}},
        ylabel={{32,}},
        fill=\PoolColor,
        height=24,
        width=2,
        depth=12
    }};

% 4. 第一个复数残差块 (64 filters)
\pic[shift={(3,0,0)}] at (complex_pool1-east) 
    {RightBandedBox={
        name=res_block1,
        caption=残差块1,
        xlabel={{64,}},
        ylabel={{32,}},
        fill=\ConvColor,
        height=24,
        width=4,
        depth=12
    }};

% 第一个残差连接 (块内连接)
\draw[residual_connection] ($(res_block1-west)+(0,0.5,0)$) 
    to[out=90,in=90] 
    ($(res_block1-east)+(0,0.5,0)$);

% 5. 第二个复数残差块 (128 filters, strides=2)
\pic[shift={(3,0,0)}] at (res_block1-east) 
    {RightBandedBox={
        name=res_block2,
        caption=残差块2,
        xlabel={{128,}},
        ylabel={{16,}},
        fill=\ConvColor,
        height=20,
        width=5,
        depth=10
    }};

% 第二个残差连接 (块内连接)
\draw[residual_connection] ($(res_block2-west)+(0,0.5,0)$) 
    to[out=90,in=90] 
    ($(res_block2-east)+(0,0.5,0)$);

% 6. 高级复数残差块 (256 filters, strides=2)
\pic[shift={(3,0,0)}] at (res_block2-east) 
    {RightBandedBox={
        name=res_block3,
        caption=高级残差块,
        xlabel={{256,}},
        ylabel={{8,}},
        fill=\ConvColor,
        height=16,
        width=6,
        depth=8
    }};

% 第三个残差连接 (块内连接)
\draw[residual_connection] ($(res_block3-west)+(0,0.5,0)$) 
    to[out=90,in=90] 
    ($(res_block3-east)+(0,0.5,0)$);

% 7. 复数全局平均池化
\pic[shift={(2.5,0,0)}] at (res_block3-east) 
    {Box={
        name=global_pool,
        caption=ComplexGAP,
        xlabel={{256,}},
        ylabel={{1,}},
        fill=\PoolColor,
        height=12,
        width=3,
        depth=2
    }};

% 8. 复数全连接层
\pic[shift={(2.5,0,0)}] at (global_pool-east) 
    {Box={
        name=complex_dense,
        caption=ComplexDense,
        xlabel={{512,}},
        fill=\FcColor,
        height=10,
        width=4,
        depth=4
    }};

% 9. 复数幅度提取
\pic[shift={(2,0,0)}] at (complex_dense-east) 
    {Box={
        name=magnitude,
        caption=幅度提取,
        xlabel={{512,}},
        fill=\ReshapeColor,
        height=8,
        width=3,
        depth=3
    }};

% 10. 实数全连接层
\pic[shift={(2,0,0)}] at (magnitude-east) 
    {Box={
        name=real_dense,
        caption=Dense,
        xlabel={{256,}},
        fill=\FcColor,
        height=6,
        width=3,
        depth=2
    }};

% 11. 分类输出
\pic[shift={(2,0,0)}] at (real_dense-east) 
    {Box={
        name=output,
        caption=Softmax,
        xlabel={{11,}},
        fill=\SoftmaxColor,
        height=4,
        width=2,
        depth=1
    }};

% 连接线
\draw[connection]  (input-east)        -- node {\midarrow} (permute-west);
\draw[connection]  (permute-east)      -- node {\midarrow} (complex_conv1-west);
\draw[connection]  (complex_conv1-east) -- node {\midarrow} (complex_bn1-west);
\draw[connection]  (complex_bn1-east)  -- node {\midarrow} (complex_pool1-west);
\draw[connection]  (complex_pool1-east) -- node {\midarrow} (res_block1-west);
\draw[connection]  (res_block1-east)   -- node {\midarrow} (res_block2-west);
\draw[connection]  (res_block2-east)   -- node {\midarrow} (res_block3-west);
\draw[connection]  (res_block3-east)   -- node {\midarrow} (global_pool-west);
\draw[connection]  (global_pool-east)  -- node {\midarrow} (complex_dense-west);
\draw[connection]  (complex_dense-east) -- node {\midarrow} (magnitude-west);
\draw[connection]  (magnitude-east)    -- node {\midarrow} (real_dense-west);
\draw[connection]  (real_dense-east)   -- node {\midarrow} (output-west);

% 阶段标注
\node[above=1cm of complex_conv1] {\Large \textbf{复数特征提取阶段}};
\node[above=1cm of res_block2] {\Large \textbf{复数残差处理阶段}};
\node[above=1cm of complex_dense] {\Large \textbf{全局特征处理阶段}};
\node[above=1cm of real_dense] {\Large \textbf{实数分类阶段}};

% 图例
\node[rectangle, draw, fill=\ConvColor, minimum width=1cm, minimum height=0.5cm] at (12,18) {};
\node[right] at (13,18) {复数卷积层};

\node[rectangle, draw, fill=\PoolColor, minimum width=1cm, minimum height=0.5cm] at (12,16.5) {};
\node[right] at (13,16.5) {池化层};

\node[rectangle, draw, fill=\FcColor, minimum width=1cm, minimum height=0.5cm] at (12,15) {};
\node[right] at (13,15) {全连接层};

\node[rectangle, draw, fill=\ReshapeColor, minimum width=1cm, minimum height=0.5cm] at (12,13.5) {};
\node[right] at (13,13.5) {特殊操作};

% 残差连接图例
\draw[residual_connection] (18,18) -- (20,18);
\node[right] at (20.2,18) {残差连接};

% 数据流维度说明
\node[below=2cm of input, align=center] {\small 
    \textbf{数据流说明:}\\
    输入: IQ数据 (2×128)\\
    → 时间序列 (128×2)\\
    → 复数特征提取\\
    → 3个残差块处理\\
    → 全局池化与分类
};

% 关键特性标注
\node[below=3.5cm of res_block2, align=center] {\small 
    \textbf{轻量级设计特点:}\\
    • 3个残差块 (64→128→256)\\
    • 纯复数域处理\\
    • 简化的残差连接\\
    • 高效的特征提取
};

\end{tikzpicture}
\end{document}
%              2) ComplexResidualBlock (2-layer) and ComplexResidualBlockAdvanced (3-layer)
%              3) Lightweight design with correct filter progression: 32→64→128→256

\documentclass[landscape, border=5pt, multi, tikz]{standalone}
\usepackage{import}
\subimport{./layers/}{init}
\usetikzlibrary{positioning}
\usetikzlibrary{3d}
\usetikzlibrary{patterns}
\usetikzlibrary{decorations.pathreplacing}

% Enhanced color scheme for different components
\def\ComplexDomainColor{rgb:cyan,6;blue,4;white,2}        % Complex domain - cyan-blue
\def\RealDomainColor{rgb:orange,6;red,3;white,2}          % Real domain - orange
\def\BasicResidualColor{rgb:purple,7;blue,3;white,2}      % Basic residual blocks - purple
\def\AdvancedResidualColor{rgb:purple,9;red,2;white,1}    % Advanced residual blocks - deeper purple
\def\PoolColor{rgb:red,5;orange,3;white,2}                % Pooling layers - red-orange
\def\TransitionColor{rgb:green,5;yellow,3;white,2}        % Transition layers - green-yellow

% Domain identifiers
\newcommand{\complexdomain}[1]{\textcolor{cyan!80!black}{\textbf{[C]}} #1}
\newcommand{\realdomain}[1]{\textcolor{orange!80!black}{\textbf{[R]}} #1}

\begin{document}
\begin{tikzpicture}

% Define connection styles
\tikzstyle{complex_flow}=[ultra thick,draw=cyan!80!black,opacity=0.8]
\tikzstyle{real_flow}=[ultra thick,draw=orange!80!black,opacity=0.8]
\tikzstyle{residual_connection}=[thick,draw=green!70!black,opacity=0.9,dashed,->]

% Background domain separation
\fill[cyan!3!white,opacity=0.3] (-1,-3) rectangle (35,3);
\fill[orange!3!white,opacity=0.3] (35,-3) rectangle (50,3);

% Domain labels
\node[above] at (17,3.5) {\Large\textbf{Complex Domain Processing}};
\node[above] at (42.5,3.5) {\Large\textbf{Real Domain Classification}};

% ========== INPUT STAGE ==========
\pic[shift={(0,0,0)}] at (0,0,0) {Box={
    name=input_iq,
    caption=Input\\(2{,}128),
    fill=\ComplexDomainColor,
    height=25,
    width=4,
    depth=50
}};

\pic[shift={(3,0,0)}] at (0,0,0) {Box={
    name=permute,
    caption=Permute\\(128{,}2),
    fill=\ComplexDomainColor,
    height=50,
    width=4,
    depth=25
}};

% ========== INITIAL COMPLEX FEATURE EXTRACTION ==========
\pic[shift={(6.5,0,0)}] at (0,0,0) {Box={
    name=initial_conv,
    caption=ComplexConv1D\\32 filters{,} k=5,
    fill=\ComplexDomainColor,
    height=48,
    width=4,
    depth=22
}};

\pic[shift={(10,0,0)}] at (0,0,0) {Box={
    name=initial_bn,
    caption=Complex\\BatchNorm,
    fill=\ComplexDomainColor,
    height=48,
    width=3.5,
    depth=20
}};

\pic[shift={(13,0,0)}] at (0,0,0) {Box={
    name=initial_activation,
    caption=Complex\\LeakyReLU,
    fill=\ComplexDomainColor,
    height=48,
    width=3.5,
    depth=18
}};

\pic[shift={(16,0,0)}] at (0,0,0) {Box={
    name=initial_pool,
    caption=Complex\\Pool (2),
    fill=\PoolColor,
    height=24,
    width=4,
    depth=16
}};

% ========== THREE RESIDUAL BLOCKS ==========
% Residual Block 1: ComplexResidualBlock(64)
\pic[shift={(20,0,0)}] at (0,0,0) {Box={
    name=resblock1,
    caption=ResidualBlock1\\ComplexResBlock\\64 filters\\(2-Layer),
    fill=\BasicResidualColor,
    height=24,
    width=5,
    depth=14
}};

% Residual Block 2: ComplexResidualBlock(128, strides=2)
\pic[shift={(26,0,0)}] at (0,0,0) {Box={
    name=resblock2,
    caption=ResidualBlock2\\ComplexResBlock\\128 filters{,} s=2\\(2-Layer),
    fill=\BasicResidualColor,
    height=18,
    width=5,
    depth=10
}};

% Residual Block 3: ComplexResidualBlockAdvanced(256, strides=2)
\pic[shift={(32,0,0)}] at (0,0,0) {Box={
    name=resblock3,
    caption=ResidualBlock3\\AdvancedResBlock\\256 filters{,} s=2\\(3-Layer),
    fill=\AdvancedResidualColor,
    height=12,
    width=5.5,
    depth=6
}};

% ========== GLOBAL POOLING AND CLASSIFICATION ==========
\pic[shift={(37.5,0,0)}] at (0,0,0) {Box={
    name=global_pool,
    caption=Complex Global\\Average Pool,
    fill=\TransitionColor,
    height=8,
    width=4,
    depth=4
}};

\pic[shift={(41,0,0)}] at (0,0,0) {Box={
    name=complex_dense,
    caption=Complex\\Dense 512,
    fill=\ComplexDomainColor,
    height=6,
    width=3.5,
    depth=3
}};

\pic[shift={(44,0,0)}] at (0,0,0) {Box={
    name=magnitude_conversion,
    caption=Complex\\Magnitude,
    fill=\TransitionColor,
    height=4,
    width=3.5,
    depth=2
}};

% ========== REAL DOMAIN CLASSIFICATION ==========
\pic[shift={(47,0,0)}] at (0,0,0) {Box={
    name=real_dense,
    caption=Dense 256\\ReLU,
    fill=\RealDomainColor,
    height=3,
    width=3.5,
    depth=1.5
}};

\pic[shift={(50,0,0)}] at (0,0,0) {Box={
    name=output_softmax,
    caption=Softmax\\11 classes,
    fill=\RealDomainColor,
    height=2,
    width=3,
    depth=1
}};

% ========== MAIN DATA FLOW CONNECTIONS ==========
\draw [complex_flow] (input_iq-east) -- (permute-west);
\draw [complex_flow] (permute-east) -- (initial_conv-west);
\draw [complex_flow] (initial_conv-east) -- (initial_bn-west);
\draw [complex_flow] (initial_bn-east) -- (initial_activation-west);
\draw [complex_flow] (initial_activation-east) -- (initial_pool-west);
\draw [complex_flow] (initial_pool-east) -- (resblock1-west);
\draw [complex_flow] (resblock1-east) -- (resblock2-west);
\draw [complex_flow] (resblock2-east) -- (resblock3-west);
\draw [complex_flow] (resblock3-east) -- (global_pool-west);
\draw [complex_flow] (global_pool-east) -- (complex_dense-west);
\draw [complex_flow] (complex_dense-east) -- (magnitude_conversion-west);
\draw [real_flow] (magnitude_conversion-east) -- (real_dense-west);
\draw [real_flow] (real_dense-east) -- (output_softmax-west);

% ========== RESIDUAL CONNECTIONS (Simple and Accurate) ==========
% ResidualBlock 1: Simple identity connection
\draw [residual_connection] (resblock1-south) to[out=270,in=270,looseness=1.5] (resblock1-north);
\node[below,font=\tiny,color=green!70!black] at ([yshift=-0.3cm]resblock1.south) {Identity: H(x) = F(x) + x};

% ResidualBlock 2: Identity connection with dimension matching
\draw [residual_connection] (resblock2-south) to[out=270,in=270,looseness=1.5] (resblock2-north);
\node[below,font=\tiny,color=green!70!black] at ([yshift=-0.3cm]resblock2.south) {Shortcut: 1×1 conv for dimension matching};

% ResidualBlock 3: Advanced 3-layer structure
\draw [residual_connection] (resblock3-south) to[out=270,in=270,looseness=1.5] (resblock3-north);
\node[below,font=\tiny,color=green!70!black] at ([yshift=-0.3cm]resblock3.south) {Bottleneck: 1×1→3×3→1×1 + shortcut};

% Data flow dimension annotations
\node[above,font=\tiny,color=purple!80!black] at ([yshift=0.5cm]resblock1.north) {(64,64)};
\node[above,font=\tiny,color=purple!80!black] at ([yshift=0.5cm]resblock2.north) {(64,64)→(128,32)};
\node[above,font=\tiny,color=purple!90!red] at ([yshift=0.5cm]resblock3.north) {(128,32)→(256,16)};

% ========== TITLE AND ANNOTATIONS ==========
\node[above] at (25,4.5) {\Huge\textbf{Lightweight Hybrid Complex-ResNet Architecture}};
\node[above] at (25,4) {\Large\textbf{简化的3残差块架构 - 轻量级设计}};
\node[above] at (25,3.6) {\normalsize\textcolor{green!70!black}{\textbf{32→64→128→256 filter progression with complex residual learning}}};

% Core architecture features explanation
\node[below,text width=14cm,align=center] at (25,-2.8) {\small\textbf{核心架构特点:} 
\\• \textcolor{purple!80!black}{ComplexResidualBlock (2层)}: Conv→BN→Act→Conv→BN→(+residual)→Act
\\• \textcolor{purple!90!red}{ComplexResidualBlockAdvanced (3层瓶颈)}: 1×1→3×3→1×1 + shortcut connection
\\• \textcolor{cyan!80!black}{Pure Complex Processing}: 从输入到全局池化全程复数域计算
\\• \textcolor{green!70!black}{简化的残差连接}: 三个独立残差块,避免复杂跨层连接
\\• \textcolor{orange!80!black}{Complex→Real转换}: ComplexMagnitude层实现域间转换};

% Input/Output labels
\node[below] at (0,-3.5) {\small\textbf{I/Q Radio Signals}};
\node[below] at (50,-3.5) {\small\textbf{11 Modulation Classes}};

% Stage annotations
\node[above,font=\small,color=cyan!80!black] at (9.75,2.8) {\textbf{初始复数特征提取}};
\node[above,font=\small,color=purple!80!black] at (20,2.8) {\textbf{ResBlock 1}};
\node[above,font=\small,color=purple!80!black] at (26,2.8) {\textbf{ResBlock 2}};
\node[above,font=\small,color=purple!90!red] at (32,2.8) {\textbf{ResBlock 3}};
\node[above,font=\small,color=green!80!black] at (42.5,2.8) {\textbf{全局池化与分类}};

% Legend - Simplified
\node[below] at (8,-4.5) {\footnotesize\textcolor{cyan!80!black}{\textbf{——}} 主数据流};
\node[below] at (18,-4.5) {\footnotesize\textcolor{green!70!black}{\textbf{- - -}} 残差连接};
\node[below] at (28,-4.5) {\footnotesize\textcolor{purple!80!black}{\textbf{■}} 基础残差块};
\node[below] at (38,-4.5) {\footnotesize\textcolor{purple!90!red}{\textbf{■}} 高级残差块};
\node[below] at (48,-4.5) {\footnotesize\textcolor{orange!80!black}{\textbf{——}} 实数域};

\end{tikzpicture}
\end{document}
